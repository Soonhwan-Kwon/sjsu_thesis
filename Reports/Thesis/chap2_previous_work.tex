\chapter{Previous Work}\label{chap:previousWork}

Computational jigsaw puzzle solvers have been studied since the 1960s when Freeman~\& Gardner~\cite{freeman1964} proposed an algorithm that could solve puzzles of up to nine pieces using only piece shape.  Since then, the focus of research has gradually shifted from traditional jigsaw puzzles to jig swap puzzles.  

In 2010, Cho~\textit{et al.}~\cite{cho2010} proposed one of the first modern, computational jig swap puzzle solvers; their approach relied on a graphical model built around a set of one or more ``anchor piece(s),'' whose position is fixed in the correct location before placement of other pieces begins.  Future solvers would improve on Cho~\textit{et al.}'s results while simultaneously reducing the amount of information (i.e., beyond the set of pieces) passed to the solver.

A significant contribution of Cho~\textit{et al.} is that they were first to use the LAB~(\underline{L}ightness and the \underline{A}/\underline{B} opponent color dimensions) colorspace to encode image pixels.  It was selected due to its property of normalizing the lightness and color variation across all three pixel dimensions.  Cho~\textit{et al.} also proposed a measure for quantifying the pairwise distance between two puzzle pieces that became the basis of most future work.  

Pomeranz~\textit{et al.}~\cite{pomeranz2011} published an iterative, greedy, jig swap puzzle solver in 2011.  Their approach did not rely on anchor pieces, and the only information passed to the solver were the pieces, their orientation, and the puzzle dimensions.  In addition, Pomeranz~\textit{et al.} introduced the concept of ``best buddies,'' which is any pair of pieces that are more compatible with each other on their respective sides than they are to any other piece.  This is formally defined in Equation~\eref{eq:pomeranzBestBuddyDefinition} for side $s_x$ (e.g., top, left, right, bottom) of puzzle piece, $p_i$, and side, $s_y$, of piece $p_j$.  $C(p_i, s_x, p_j, s_y)$ represents the compatibility between the two pieces' respective sides.

\begin{equation} \label{eq:pomeranzBestBuddyDefinition}
\centering
\begin{split}
	\begin{matrix}
		\forall{p_{k}}\forall{s_z},C(p_i, s_x, p_j, s_y) \geq C(p_i, s_x, p_k, s_z)
		\\
		\\
		\textnormal{and}
		\\
		\\
		\forall{p_{k}}\forall{s_z},C(p_j, s_y, p_i, s_x) \geq C(p_j, s_y, p_k, s_z)
	\end{matrix}
\end{split}
\end{equation} 

The best buddies relationship has served as both a metric for estimating the quality of a solver output~\cite{sholomon2013} as well as the foundation of some solvers' assemblers~\cite{paikin2015}.  Best buddies are discussed extensively in Sections~\ref{sec:segmentPuzzle},~\ref{sec:bestBuddyMetrics}, and~\ref{sec:singlePuzzleSolving} of this thesis.

An additional key contribution of Pomeranz~\textit{et al.} is the creation of three image benchmarks.  The first benchmark is comprised of twenty, 805 piece images; this benchmark is used as the test set for the experiments described in Chapter~\ref{chap:experimentalResults}. There are three images in each of the other two benchmarks, with images in the first data set having 2,360 pieces while those in the other benchmark have 3,300 pieces.

In 2012, Gallagher~\cite{gallagher2012} formally categorized jig swap puzzle problems into four primary types.  The following is Gallagher's proposed terminology; his nomenclature is used throughout this thesis.

\begin{itemize}

	\item \textbf{Type~1 Puzzle}: The dimensions of the puzzle (i.e., the width and height of the ground-truth image in number of pixels) is known.  The orientation/rotation of each piece is also known, which means that there are exactly four pairwise relationships between any two pieces.  In addition, the solver may be provided with the correct location of one or more ``anchor'' pieces.  This type of puzzle is the focus of~\cite{cho2010, pomeranz2011}.
	
	\item \textbf{Type~2 Puzzle}: This is an extension of the Type~1 puzzle, where pieces may be rotated in \numbwithdegreesymbol{90} increments (e.g., \numbwithdegreesymbol{0}, \numbwithdegreesymbol{90}, \numbwithdegreesymbol{180}, or \numbwithdegreesymbol{270}); in comparison to Type~1, this change alone increases the number of possible solutions by a factor of $4^n$, where $n$ is the number of puzzle pieces.  Additionally, all piece locations are unknown, which means there are no anchor pieces.  Lastly, the dimensions of the ground-truth image may be unknown.
	
	\item \textbf{Type 3 Puzzle}: All puzzle piece locations are known, and only the rotation of the pieces is unknown.  This is the least computationally complex of the puzzle variants and is generally considered the least interesting.  Type 3 puzzles are not explored as part of this thesis.
	
	\item \textbf{Mixed-Bag Puzzle}: The input set of pieces are from multiple puzzles.  The solver may output either a single, merged puzzle, or it may separate the puzzle pieces into disjoint sets that ideally align with the set of ground-truth input images.  This type of puzzle is the primary focus of this thesis.

\end{itemize}

In 2013, Sholomon~\textit{et al.}~\cite{sholomon2013} presented a genetic algorithm-based solver for Type~1 puzzles.  By moving away from the greedy paradigm used by Pomeranz~\textit{et al.}, Sholomon~\textit{et al.}'s approach is more immune to suboptimal decisions early in the placement process. Sholomon~\textit{et al.}'s algorithm is able to solve puzzles of significantly larger size than other techniques (e.g., greater than 23,000 pieces).  What is more, Sholomon~\textit{et al.} defined three new large image benchmarks; the specific puzzle sizes are 5,015, 10,375, and 22,834 pieces~\cite{sholomonBenchmarkImages}.

Paikin~\& Tal~\cite{paikin2015} introduced in 2015 a greedy solver that handles both Type~1 and Type~2 puzzles, even if those puzzles are missing pieces.  What is more, their algorithm is one of the first to support Mixed-Bag Puzzles.  While Paikin~\&~Tal's algorithm represents the current state of the art, it has serious limitations.  For example, similar to previous solvers, Paikin~\&~Tal's algorithm must be told the number of input puzzles.  In many practical applications, this information may not be known.

Another limitation arises from the fact that Paikin~\& Tal's algorithm places pieces using a single-pass, kernel growing approach.  As such, a single piece is used as the seed of each output puzzle, and all subsequent pieces are placed around the expanding kernel.  If a seed is selected poorly, the quality of the solver output may be catastrophically degraded.  Despite this, their algorithm only requires that a seed piece have best buddies on each of its sides and that the seed's best buddies also have best buddies on each of their sides.  Therefore, the selection of the seed is based off essentially 13~pieces.  What is more, the selection of the seed is performed greedily at run time.  Through the combination of these two factors, it is common that the seeds of multiple output puzzles come from the same ground-truth image.

The limitations of Paikin~\&~Tal's algorithm are addressed by this thesis' Mixed-Bag Solver, which is described in Chapter~\ref{chap:mixedBagSolver}.  Since Paikin \&  Tal's algorithm represents the current state of the art, it is used as this thesis' assembler.  What is more, their algorithm is used as the baseline for all performance comparisons. 