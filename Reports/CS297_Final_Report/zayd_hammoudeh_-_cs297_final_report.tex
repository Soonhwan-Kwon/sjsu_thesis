\documentclass{report}

\usepackage{url}
\usepackage{indentfirst}
\usepackage{float}

\usepackage[T1]{fontenc}
\usepackage{textcomp} % Required for upquote.
\usepackage{listings} % Include the listings-package
% Ensure quotes in listings are straight.
% Cleaner way to print strings in listings packages so no space symbol.
\lstset{showstringspaces=false, 
        upquote=true} 


\usepackage{mdframed}
\usepackage[numbers,sort]{natbib}
%\usepackage[english]{babel} % Need for text wrap in table.
\usepackage{array} % Needed for centering in the table
\usepackage[export]{adjustbox} % loads also graphicx
\usepackage{graphicx}

\usepackage{hyperref} % Creates links in the PDF document.
\hypersetup{hidelinks} % Do not include boxes around links

% Defines the table of contents depth and the subsection numbering depth
\setcounter{secnumdepth}{5}
\setcounter{tocdepth}{5}

\title{An Enhanced Jigsaw Puzzle Solver \\[1in]
	   CS297 Final Report}

\author{
  Zayd Hammoudeh \\
  (zayd.hammoudeh@sjsu.edu)
  }


\newcommand{\myparagraph}[1]{\paragraph{#1}\mbox{}\\}

% Skip lines after each paragraph.
\setlength\parskip{\baselineskip}

\begin{document}

\maketitle

\pagenumbering{roman}

\tableofcontents{\protect\newpage}

\addcontentsline{toc}{section}{List of Figures}
\listoffigures
\newpage
 
\pagenumbering{arabic}

\renewcommand\thesection{\arabic{section}}

\section{Introduction}\label{sec:introduction}

Jigsaw puzzles have been around since the 1760s when they were made from wood.  Their name derives from the fact that they were originally carved using jigsaws.   The 1930s saw the introduction of the modern jigsaw puzzle where an image was printed on a cardboard sheet that was cut into a set of interlocking pieces.  Although jigsaw puzzles had been solved by children for centuries, it was not until 1964 that the first automated jigsaw puzzle solver was proposed by \cite{freeman1964}, and that solver could only solve 9 piece puzzles.  While an automated jigsaw puzzle solver may seem trivial, it has been shown by \cite{altman1990} and \cite{demaine2007} to be strongly NP-complete when pairwise compatibility between pieces is not a reliable metric for determining adjacency.

A jig swap puzzles are specific type of jigsaw puzzle where  all pieces are equally sized, non-overlapping squares.  Jig swap puzzles are substantially more difficult to solve than standard jigsaw puzzle as one can not consider mechanical compatibility when trying to determine affinity between pieces.  As such, one can only consider the image information on each individual piece when solving the puzzle.  

Solving a jigsaw puzzle simplifies to reconstructing an object from a set of component pieces.  As such, techniques developed for jigsaw puzzles can be generalized to many practical problems.  Examples where jigsaw puzzle solving strategies are applicable include: reassembly of archaeological artifacts \cite{brown2008, koller2006}, forensic analysis of deleted files \cite{garfinkel2010}, image editing \cite{cho2008}, reconstruction of shredded documents \cite{zhu2008}, DNA fragment reassembly \cite{marande2007}, and speech descrambling \cite{zhao2007}.

Unlike traditional jigsaw or jig swap puzzles, the original (i.e. target) image is unknown for most practical applications.  This significantly complicates solving the problem as one must determine the overall structure of the complete solution solely from a bag of individual pieces with unknown relationships.

This project proposes an improved jig swap puzzle solver.  It also studies the fundamental structures within images to determine which images are easy or difficult to solve using current techniques.  This study will help guide future research by focusing on the weaknesses with state of the art approaches. 

\pagebreak
\section{Previous Work}\label{sec:previousWork}

Computational solvers for jigsaw puzzles have been studied since the 1960's when Freeman and Garder proposed an approach that could solve jigsaw puzzles of up-to nine pieces using solely the piece shapes \cite{freeman1964}.  Since then the focus of research has shifted from traditional jigsaw puzzles to jig swap puzzles.  

\pagebreak
\bibliographystyle{ieeetr}
\bibliography{cs297_final_report_biblio}

\end{document}
